%%
%% This is file `sample-sigconf.tex',
%% generated with the docstrip utility.
%%
%% The original source files were:
%%
%% samples.dtx  (with options: `sigconf')
%% 
%% IMPORTANT NOTICE:
%% 
%% For the copyright see the source file.
%% 
%% Any modified versions of this file must be renamed
%% with new filenames distinct from sample-sigconf.tex.
%% 
%% For distribution of the original source see the terms
%% for copying and modification in the file samples.dtx.
%% 
%% This generated file may be distributed as long as the
%% original source files, as listed above, are part of the
%% same distribution. (The sources need not necessarily be
%% in the same archive or directory.)
%%
%%%% Proceedings format for most of ACM conferences (with the exceptions listed below) and all ICP volumes.
\documentclass[sigconf,nonacm,11pt]{acmart}
%%%% As of March 2017, [siggraph] is no longer used. Please use sigconf (above) for SIGGRAPH conferences.

%%%% Proceedings format for SIGPLAN conferences 
% \documentclass[sigplan, anonymous, review]{acmart}

%%%% Proceedings format for SIGCHI conferences
% \documentclass[sigchi, review]{acmart}

%%%% To use the SIGCHI extended abstract template, please visit
% https://www.overleaf.com/read/zzzfqvkmrfzn

%%
%% \BibTeX command to typeset BibTeX logo in the docs
\AtBeginDocument{%
  \providecommand\BibTeX{{%
    \normalfont B\kern-0.5em{\scshape i\kern-0.25em b}\kern-0.8em\TeX}}}

\graphicspath{{fig/}{./}}

%%TC:ignore
%% Rights management information.  This information is sent to you
%% when you complete the rights form.  These commands have SAMPLE
%% values in them; it is your responsibility as an author to replace
%% the commands and values with those provided to you when you
%% complete the rights form.
\copyrightyear{2020}
\acmYear{2019}
\setcopyright{rightsretained}

%% These commands are for a PROCEEDINGS abstract or paper.
\acmConference{CSE6242}
\acmDOI{Data and Visual Analytics}
\acmISBN{}
\acmBooktitle{}
%%TC:endignore

%%
%% Submission ID.
%% Use this when submitting an article to a sponsored event. You'll
%% receive a unique submission ID from the organizers
%% of the event, and this ID should be used as the parameter to this command.
%%\acmSubmissionID{123-A56-BU3}

%%
%% The majority of ACM publications use numbered citations and
%% references.  The command \citestyle{authoryear} switches to the
%% "author year" style.
%%
%% If you are preparing content for an event
%% sponsored by ACM SIGGRAPH, you must use the "author year" style of
%% citations and references.
%% Uncommenting
%% the next command will enable that style.
%%\citestyle{acmauthoryear}

%%
%% end of the preamble, start of the body of the document source.
\begin{document}

%%
%% The "title" command has an optional parameter,
%% allowing the author to define a "short title" to be used in page headers.
% \font\helvetica=cmr12 at 40pt
% \title{{\helvetica Who knows}}
% \author{{\small A.~U.~Thor}}

\subtitle{Socioconomic Status and the Impact on Academic Performance}

%%
%% The "author" command and its associated commands are used to define
%% the authors and their affiliations.
%% Of note is the shared affiliation of the first two authors, and the
%% "authornote" and "authornotemark" commands
%% used to denote shared contribution to the research.

%%TC:ignore
\author{Dave Dyer}
\affiliation{%
  \institution{\small Georgia Institute of Technology}}

\author{Nick Orangio}
\affiliation{%
  \institution{\small Georgia Institute of Technology}}

\author{Ben Spivey}
\affiliation{%
  \institution{\small Georgia Institute of Technology}}

\author{Kshitij Srivastava}
\affiliation{%
 \institution{\small Georgia Institute of Technology}}

\author{Vuong Tran}
\affiliation{%
  \institution{\small Georgia Institute of Technology}}



%%
%% By default, the full list of authors will be used in the page
%% headers. Often, this list is too long, and will overlap
%% other information printed in the page headers. This command allows
%% the author to define a more concise list
%% of authors' names for this purpose.
%%TC:ignore
%%TC:endignore

%%
%% The abstract is a short summary of the work to be presented in the
%% article.


%%
%% The code below is generated by the tool at http://dl.acm.org/ccs.cfm.
%% Please copy and paste the code instead of the example below.
%%
%%TC:ignore
% \begin{CCSXML}
% <ccs2012>
%  <concept>
%   <concept_id>10010520.10010553.10010562</concept_id>
%   <concept_desc>Computer systems organization~Embedded systems</concept_desc>
%   <concept_significance>500</concept_significance>
%  </concept>
%  <concept>
%   <concept_id>10010520.10010575.10010755</concept_id>
%   <concept_desc>Computer systems organization~Redundancy</concept_desc>
%   <concept_significance>300</concept_significance>
%  </concept>
%  <concept>
%   <concept_id>10010520.10010553.10010554</concept_id>
%   <concept_desc>Computer systems organization~Robotics</concept_desc>
%   <concept_significance>100</concept_significance>
%  </concept>
%  <concept>
%   <concept_id>10003033.10003083.10003095</concept_id>
%   <concept_desc>Networks~Network reliability</concept_desc>
%   <concept_significance>100</concept_significance>
%  </concept>
% </ccs2012>
% \end{CCSXML}

% \ccsdesc[500]{Computer systems organization~Embedded systems}
% \ccsdesc[300]{Computer systems organization~Redundancy}
% \ccsdesc{Computer systems organization~Robotics}
% \ccsdesc[100]{Networks~Network reliability}

%%
%% Keywords. The author(s) should pick words that accurately describe
%% the work being presented. Separate the keywords with commas.
% \keywords{education, socioeconomic status, machine learning }
%%TC:endignore

%% A "teaser" image appears between the author and affiliation
%% information and the body of the document, and typically spans the
%% page.
% \begin{teaserfigure}
%   \includegraphics[width=\textwidth]{sampleteaser}
%   \caption{Seattle Mariners at Spring Training, 2010.}
%   \Description{Enjoying the baseball game from the third-base
%   seats. Ichiro Suzuki preparing to bat.}
%   \label{fig:teaser}
% \end{teaserfigure}

%%
%% This command processes the author and affiliation and title
%% information and builds the first part of the formatted document.
%%TC:ignore
\maketitle
%%TC:endignore



\section{Introduction \& Objectives (HQ1)}

The impact of  social \& economic status (SES) on students' academic performance (AP) is a well-studied area of research.  Academic performance is often positively correlated with higher resources/funding \cite{jinnai}and negatively correlated with higher needs, such as free and reduced lunch programs\cite{sirin}, but the magnitude of the correlation is not widely agreed-upon. In this project, we intend to model the impact of SES on AP using New York State School data. We will produce a working model and an interactive visual interface to explore the model interactively. Like many topics with sufficient data, this area is well studied. We seek to improve on status quo by investigating additional confounding factors (e.g., teacher education), evaluating Machine Learning (ML) models compared to state of the art fixed-effect \& statistical models \cite{hearn, farooq}, and creating insightful, interactive data visualizations. Applying both ML and thoughtful visualization design should have an impact on this area of research as we aim to improve on the state of the art.

\section{Prior Work \& Current Practices (HQ2)}
%How is it done today, and what are the limits of current practice? 

The relationship between SES and academic achievement has been explored earlier in Benner et al. \cite{parentalinvolvement}, Zwick \cite{collegescores} and Battle et al. \cite{raceses}. All prior work agree in differing measures that SES is an important predictor of academic achievement in combination with other factors like parental involvement \cite{parentalinvolvement}, sytematic test scores, ethnic group \cite{collegescores} and race \cite{raceses}. We've also seen that \cite{parentalinvolvement, collegescores, raceses, farooq}, most of prior research relies on parametric models where there is an attempt to quantify said relationships by evaluating relative goodness of fits using ANOVA and T-Tests. While this is a good practice to understand relative relationships like in \cite{collegescores, raceses}, it fails to capture random effects and confounding factors. We learned that linear fixed effects models are commonly used to account for confounding factors for educational studies \cite{winters, jinnai}. Our work will use a number of machine learning based models to capture any non-linear relationships. Additionally, \cite{collegescores} also has a possible scenario of Simpson's paradox which we will try to avoid. Typically, \cite{sirin} SES is measured by a few different measures -- Free \& Reduced Lunch Ratio (FRL), family education levels, Occupation, and Income.  Academic performance, meanwhile, is typically measured by GPA or some form of standardized testing \cite{sirin}.  We intend to explore the relationship between other, more novel, aggregate measures of both SES and academic performance. Finally, we intend to provide an intuitive and interactive graphic interface to explore results.

\section{Approach (HQ3)}
%What is new in your approach and why do you think it will be successful? 
From the graduation rate dataset at  https://data.nysed.gov/, we intend to test multiple co-variates including lurking variables like school types \cite{west} at the school, neighborhood, district, and state level, in order to characterize what constitutes SES and academic performance at various levels of aggregation . We'll use a graphing library (D3) to allow the user to explore various segments of data in both a univariate and bivariate manner. We seek to find correlations, not causal relationships, as the data may not be sufficient to include all latent factors. We will explore both machine learning and traditional statistical learning algorithms to characterize relationships between SES and academic performance at differing levels of aggregation. We also plan to investigate confounding factors that could affect conclusions on SES, such as school types and teacher education. We will also explore using fixed effect models as used in economic studies of charter school factors \cite{winters, jinnai}. We may need to perform feature engineering to define the factors appropriately and must consider nuance as some factors like charter schools may only affect grades levels (e.g., 6th grade) that are overlapping \cite{jinnai}.

\section{Stakeholders and Impact (HQ4 \& HQ5)}
%Who cares? If you are successful, what difference will it make? 
Some surprise beneficiaries of our research are people interested in healthier lifestyles for students, based on J.O. Lee's research \cite{lee}, which links poor academic performance with worse health outcomes. Additionally, researchers of the relationship between school climate, SES, and Academic performance would benefit; In Berkowitz's study of school climate impact on AP she noted a lack of consistent SES - AP modeling \cite{berkowitz}. Last, this project should be of interest to people invested in improving academic performance overall. While school administrators often point to overall improving academic performance \cite{domanico}, there is broad consensus that the SES - AP correlation is mitigated by well-resourced schools (such as charter schools) \cite{domanico, jinnai}. If we are successful, all of the above could use our research to help reinforce their arguments for improving resources for lower SES schools.

\section{Risks (HQ6)}
%What are the risks?
In Sirin, et. al.'s Review of Educational Research, one of the primary risks is 'ecological fallacy' \cite{sirin} -- a misinterpretation of the results where one applies aggregate results to individual outcomes. This very much applies to our project, since we work with aggregate data at the school, district, county, and state level.  In order to not run afoul of this risk, we will need to clearly indicate, visually, what level we are operating at and reinforce that outcomes may not apply to individuals.  Another risk that data scientists should consider is the 'black box' problem of machine learning models. Gilpin claims that complex machines and algorithms cannot perform interpretation tasks \cite{gilpin}.  If our best models happen to be ML models we run the risk of not being able to explain how the model arrives at its decisions. Another risk is that the machine learning models might downplay important features if the features are highly correlated. We will need to examine feature correlation before making conclusions on feature importance.

\section{Cost HQ7}
%How much will it cost? 
Based on How to Get a Free Lunch: A simple cost model for machine learning applications, a straightforward cost model for ML applications can be applied. \cite{domingos} This somewhat applies to our project, because while we hope to not incur additional equipment, storage, and compute costs, it is possible that our scope expands to where only distributed computing will be able to handle the calculations.  The cost for an ML project is the following:\\
$NPV = C_0 + \Sigma{\frac{C_t}{(1+r)^t}}$\\
Where $NPV$ is the net present value, $C_0$ is the initial cost of equipment, and $C_t$ is the decision cost. Since we are working with freely available data, and we intend to use compute \& storage we already own, the NPV of this project should be \$0.


\section{Time (HQ8)}
%How long will it take? 
In Kushal Singla, et. al's analysis \cite{mltime} of software engineering for Agile ML projects, they identify that a slight majority of tasks end up in the backlog for ML projects, and there is a huge time variance on how long a story takes. On average, an ML story takes about 18 hours to complete, with an extremely high standard deviation of 31 hours.  For our team of 5, we intend to divide and conquer many tasks with Nick and Dave working more heavily on the documentation, design, organization, presentation, and oversight; Vuong and Ben working more heavily on D3 visualization and python modeling, respectively; and Kshitij working on a pretty even blend of all of the above.  We intend to adopt an agile approach, and remain flexible as new information / requirements arise.  We estimate 1 week for lit survey, proposal, and presentation development.  We estimate 5 stories (roughly 90±155h) in a two week sprint for data cleaning, loading, design, EDA, and preliminary modeling; another 5 (90±155h) stories for Statistics/ ML modeling, and 8 stories (144±248h) for D3 visualization, interactivity, testing and iteration.

\section{Success Criteria \& Evaluation (HQ9)}
%What are the mid-term and final “exams” to check for success? 
At a project level, our Minimum Viable Product (MVP) is a working, well-researched model and a working interactive data exploratory visualization tool. We intend to evaluate the model using metrics that will vary, depending on what models we choose to compare.  At the bare minimum, we will use AUC or F1 Score for the model. We will know if we are successful if we can get data cleaned, model developed, and a Choropleth map working by the end of October as an MVP.

%P-values, F scores, working visualization \& MVP.  We are shooting for a minimum of two-model comparisons and a 40-60\% blend of modeling to visualization focus ratio.

\begin{thebibliography}{9}
\bibitem{jinnai}
Jinnai, Y. (2014). "Direct and indirect impact of charter schools’ entry on traditional public schools: New evidence from North Carolina." \textit{Economic Letters}, 124, 452-456.

\bibitem{sirin}
Sirin, S. R. (2005).
"Socioeconomic Status and Academic Achievement: A Meta-Analytic Review of Research." \textit{Review of Educational Research}, 75(3), 417–453.

\bibitem{hearn}
Hearn, J. C. (1988). "Attendance at Higher-Cost Colleges: Ascribed, Socioeconomic, and Academic Influences on Student Enrollment Patterns." \textit{Economics of Education Review},7(1), 65-76.

\bibitem{farooq}
Farooq, M.S., Chaudhry, A.H., Shafiq, M., Berhanu, G. (2011). "Factors Affecting Students' Quality of Academic Performance: A Case of Secondary School Level." \textit{Journal of Quality and Technology Management}, VII(II)01-04.

\bibitem{parentalinvolvement} 
Benner, A. D., Boyle, A. E., Sadler, S. (2016). "Parental Involvement and Adolescents’ Educational Success: The Roles of Prior Achievement and Socioeconomic Status." \textit{Journal of Youth and Adolesence}, 45, 1053-1064.

\bibitem{collegescores} 
Zwick, R. (2012). "The Role of Admissions Test Scores, Socioeconomic Status, and High School Grades in Predicting College Achievement." \textit{Pensamiento Educativo. Revista de Investigación Educacional Latinoamericana}, 49(2), 23-30.

\bibitem{raceses} 
Battle, J., Lewis, M. (2008). "The Increasing Significance of Class: The Relative Effects of Race and Socioeconomic Status on Academic Achievement." \textit{Journal of Poverty}, 6(2), 21-35.

\bibitem{lee} 
Lee, J. O., Kosterman, R., Jones, T.M., Herrenkohl, T.I., Rhew, I.C., Catalano, R.F., Hawkins, J.D. (2016) "Mechanisms linking high school graduation to health disparities in young adulthood: a longitudinal analysis of the role of health behaviours, psychosocial stressors, and health insurance." \textit{Public Health}, 139, 61-69.

\bibitem{gilpin}
Gilpin, L. H., Bau, D., Yuan, B. Z., Bajwa, A., Specter,  M., Kagal, L. (2018). \textit{Explaining Explanations: An Overview of Interpretability of Machine Learning.} IEEE 5th International Conference on Data Science and Advanced Analytics (DSAA), Turin, Italy, 80-89.

\bibitem{domanico}
Domanico, Ray (2020). "NYC Student Achievement: What State and National Test Scores Reveal."  \textit{The Manhattan Institute for Policy Research} https://eric.ed.gov/?id=ED604331. 

\bibitem{mltime}
Singla, K., Bose, J., Naik, C. (2018) \textit{Analysis of Software Engineering for Agile Machine Learning Projects.} 15th IEEE India Council International Conference (INDICON), Coimbatore, India, 1-5, doi: 10.1109/INDICON45594.2018.8987154.

\bibitem{berkowitz}
Berkowitz, R., Moore, H., Astor, R.A., Benbenishty, R. (2017). "A Research Synthesis of the Associations Between Socioeconomic Background, Inequality, School Climate, and Academic Achievement" \textit{Review of Educational Research}, 87(2), 425–469.

\bibitem{winters}
Winters, M. (2012). "Measuring the effect of charter schools on public school student achievement in an urban environment: Evidence from New York City." \textit{Economics of Education Review}, 31, 293-301.

\bibitem{west}
West, M. (2016). "Schools of choice: expanding opportunity for urban minority students." \textit{Education Next}, 16(2).

\bibitem{domingos}
Domingos, P. (1998). \textit{How to Get a Free Lunch: A Simple Cost Model for Machine Learning Applications.} 15th National Conference Artificial Intelligence (AAAI-98) and International Conference on Machine Learning (ICML-98): Workshop on AI Approaches to Time Series Problems.

\end{thebibliography}
% %%
% %% The acknowledgments section is defined using the "acks" environment
% %% (and NOT an unnumbered section). This ensures the proper
% %% identification of the section in the article metadata, and the
% %% consistent spelling of the heading.
% \begin{acks}
% To Robert, for the bagels and explaining CMYK and color spaces.
% \end{acks}

%%
%% The next two lines define the bibliography style to be used, and
%% the bibliography file.
\bibliographystyle{ACM-Reference-Format}
\bibliography{sample-base}

%%
%% If your work has an appendix, this is the place to put it.
\appendix

\end{document}
\endinput
%%
%% End of file `sample-sigconf.tex'.
