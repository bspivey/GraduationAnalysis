%%
%% This is file `sample-sigconf.tex',
%% generated with the docstrip utility.
%%
%% The original source files were:
%%
%% samples.dtx  (with options: `sigconf')
%% 
%% IMPORTANT NOTICE:
%% 
%% For the copyright see the source file.
%% 
%% Any modified versions of this file must be renamed
%% with new filenames distinct from sample-sigconf.tex.
%% 
%% For distribution of the original source see the terms
%% for copying and modification in the file samples.dtx.
%% 
%% This generated file may be distributed as long as the
%% original source files, as listed above, are part of the
%% same distribution. (The sources need not necessarily be
%% in the same archive or directory.)
%%
%%%% Proceedings format for most of ACM conferences (with the exceptions listed below) and all ICPS volumes.
\documentclass[sigconf,nonacm,11pt]{acmart}
%%%% As of March 2017, [siggraph] is no longer used. Please use sigconf (above) for SIGGRAPH conferences.

%%%% Proceedings format for SIGPLAN conferences 
% \documentclass[sigplan, anonymous, review]{acmart}

%%%% Proceedings format for SIGCHI conferences
% \documentclass[sigchi, review]{acmart}

%%%% To use the SIGCHI extended abstract template, please visit
% https://www.overleaf.com/read/zzzfqvkmrfzn

%%
%% \BibTeX command to typeset BibTeX logo in the docs
\AtBeginDocument{%
  \providecommand\BibTeX{{%
    \normalfont B\kern-0.5em{\scshape i\kern-0.25em b}\kern-0.8em\TeX}}}

\graphicspath{{fig/}{./}}

%%TC:ignore
%% Rights management information.  This information is sent to you
%% when you complete the rights form.  These commands have SAMPLE
%% values in them; it is your responsibility as an author to replace
%% the commands and values with those provided to you when you
%% complete the rights form.
\copyrightyear{2020}
\acmYear{2019}
\setcopyright{rightsretained}

%% These commands are for a PROCEEDINGS abstract or paper.
\acmConference{CSE6242}
\acmDOI{Data and Visual Analytics}
\acmISBN{}
\acmBooktitle{}
%%TC:endignore

%%
%% Submission ID.
%% Use this when submitting an article to a sponsored event. You'll
%% receive a unique submission ID from the organizers
%% of the event, and this ID should be used as the parameter to this command.
%%\acmSubmissionID{123-A56-BU3}

%%
%% The majority of ACM publications use numbered citations and
%% references.  The command \citestyle{authoryear} switches to the
%% "author year" style.
%%
%% If you are preparing content for an event
%% sponsored by ACM SIGGRAPH, you must use the "author year" style of
%% citations and references.
%% Uncommenting
%% the next command will enable that style.
%%\citestyle{acmauthoryear}

%%
%% end of the preamble, start of the body of the document source.
\begin{document}

%%
%% The "title" command has an optional parameter,
%% allowing the author to define a "short title" to be used in page headers.
\title{SESIoN: Socio Economic Status Impact on New York State Academic Performance}

%%
%% The "author" command and its associated commands are used to define
%% the authors and their affiliations.
%% Of note is the shared affiliation of the first two authors, and the
%% "authornote" and "authornotemark" commands
%% used to denote shared contribution to the research.

%%TC:ignore
\author{Dave Dyer}
%\authornotemark[1]
\affiliation{%
  \institution{Georgia Institute of Technology}
 %\streetaddress{P.O. Box 1212}
  %\city{Dublin}
  %\state{Ohio}
  %\postcode{43017-6221}
}
\email{dave.dyer@gatech.edu}


\author{Nick Orangio}
\affiliation{%
  \institution{Georgia Institute of Technology}}
 \email{norangio3@gatech.edu}

  %\streetaddress{1 Th{\o}rv{\"a}ld Circle}
 % \city{Hekla}
  %\country{Iceland}}


\author{Ben Spivey}
\affiliation{%
  \institution{Georgia Institute of Technology}}
%  \city{Rocquencourt}
%  \country{France}
 \email{gte146u@gatech.edu}


\author{Kshitij Srivastava}
\affiliation{%
 \institution{Georgia Institute of Technology}}
% \streetaddress{Rono-Hills}
% \city{Doimukh}
% \state{Arunachal Pradesh}
% \country{India}}
\email{ksrivastava34@gatech.edu}


\author{Vuong Tran}
\affiliation{%
  \institution{Georgia Institute of Technology}}
 \email{vtran62@gatech.edu}

%  \streetaddress{30 Shuangqing Rd}
%  \city{Haidian Qu}
%  \state{Beijing Shi}
%  \country{China}}

% \author{John Smith}
% \affiliation{\institution{The Th{\o}rv{\"a}ld Group}}
% \email{jsmith@affiliation.org}

% \author{Julius P. Kumquat}
% \affiliation{\institution{The Kumquat Consortium}}
% \email{jpkumquat@consortium.net}
%%TC:endignore

%%
%% By default, the full list of authors will be used in the page
%% headers. Often, this list is too long, and will overlap
%% other information printed in the page headers. This command allows
%% the author to define a more concise list
%% of authors' names for this purpose.
%%TC:ignore
%%TC:endignore

%%
%% The abstract is a short summary of the work to be presented in the
%% article.


%%
%% The code below is generated by the tool at http://dl.acm.org/ccs.cfm.
%% Please copy and paste the code instead of the example below.
%%
%%TC:ignore
% \begin{CCSXML}
% <ccs2012>
%  <concept>
%   <concept_id>10010520.10010553.10010562</concept_id>
%   <concept_desc>Computer systems organization~Embedded systems</concept_desc>
%   <concept_significance>500</concept_significance>
%  </concept>
%  <concept>
%   <concept_id>10010520.10010575.10010755</concept_id>
%   <concept_desc>Computer systems organization~Redundancy</concept_desc>
%   <concept_significance>300</concept_significance>
%  </concept>
%  <concept>
%   <concept_id>10010520.10010553.10010554</concept_id>
%   <concept_desc>Computer systems organization~Robotics</concept_desc>
%   <concept_significance>100</concept_significance>
%  </concept>
%  <concept>
%   <concept_id>10003033.10003083.10003095</concept_id>
%   <concept_desc>Networks~Network reliability</concept_desc>
%   <concept_significance>100</concept_significance>
%  </concept>
% </ccs2012>
% \end{CCSXML}

% \ccsdesc[500]{Computer systems organization~Embedded systems}
% \ccsdesc[300]{Computer systems organization~Redundancy}
% \ccsdesc{Computer systems organization~Robotics}
% \ccsdesc[100]{Networks~Network reliability}

%%
%% Keywords. The author(s) should pick words that accurately describe
%% the work being presented. Separate the keywords with commas.
\keywords{education, socioeconomic status, machine learning }
%%TC:endignore

%% A "teaser" image appears between the author and affiliation
%% information and the body of the document, and typically spans the
%% page.
% \begin{teaserfigure}
%   \includegraphics[width=\textwidth]{sampleteaser}
%   \caption{Seattle Mariners at Spring Training, 2010.}
%   \Description{Enjoying the baseball game from the third-base
%   seats. Ichiro Suzuki preparing to bat.}
%   \label{fig:teaser}
% \end{teaserfigure}

%%
%% This command processes the author and affiliation and title
%% information and builds the first part of the formatted document.
%%TC:ignore
\maketitle
%%TC:endignore



\section{Introduction \& Objectives}

The impact of  social \& economic status (SES) on students' academic performance (AP) is a well-studied area of research.  Academic performance is often positively correlated with higher resources/funding \cite{?}and negatively correlated with higher needs, such as free and reduced lunch programs\cite{sirin}, but the magnitude of the correlation is not widely agreed-upon. In this project, we intend to model the impact of SES on AP using New York State School data. We will produce a working model and an interactive visual interface to explore the model interactively. While this is a deeply-studied area, we believe that we can improve on current state of the art models, which typically use fixed-effect \& statistical models \cite{hearn, farooq, winters, jinnai}, yet eschew the use of Machine Learning (ML) models or elegant, interactive data visualization. While ML or data vizualization alone may not improve upon the state of the art, we believe that applying both ML and thoughtful visualization design can have an impact on this area of research.

% on previous research by developing visualizations that enable exploring specific correlations and factors to investigate how socioeconomic status (SES) is related to academic performance. We also plan to incorporate possible confounding factors such as school types and teacher education levels in our study which may or may not be be correlated with funding or higher needs. In addition, we plan to compare machine learning models not commonly used in our literature review to more common models in this area of research, such as classical statistical models or fixed-effect models. Combining machine learning and data visualization may provide additional insight into the relationship between SES and academic performance.

% We intend to model the impact of SES on academic performance using a novel approach and to produce a visual interface to explore the relationship between these two things interactively.  We intend to examine what SES factors have the most influence on academic performance and to examine novel measures for academic performance.  The state of the art appears not to use current data visualization and machine learning methods, and we plan to provide a better understanding of these factors through applied machine learning and thoughtful interactive data visualization. 

\section{Prior Work \& Current Practices}
%How is it done today, and what are the limits of current practice? 
The relationship between SES and academic achievement has been explored earlier in Benner et al. \cite{parentalinvolvement}, Zwick \cite{collegescores} and Battle et al. \cite{raceses}. All prior work agree in differing measures that SES is an important predictor of academic achievement in combination with other factors like parental involvement \cite{parentalinvolvement}, sytematic test scores, ethnic group \cite{collegescores} and race \cite{raceses}. We've also seen that \cite{parentalinvolvement, collegescores, raceses, farooq}, most of prior research relies on parametric models where there is an attempt to quantify said relationships by evaluating relative goodness of fits using ANOVA and T-Tests. While this is a good practice to understand relative relationships like in \cite{collegescores, raceses}, it fails to capture random effects and confounding factors. Some authors have used linear fixed effects models which account for confounding factors \cite{winters, jinnai}. Our work will use a number of machine learning based models to capture any non-linear relationships. Additionally, \cite{collegescores} also has a possible scenario of Simpson's paradox which we will try to avoid. Typically, \cite{sirin} SES is measured by a few different measures -- Free \& Reduced Lunch Ratio (FRL), family education levels, Occupation, and Income.  Academic performance, meanwhile, is typically measured by GPA or some form of standardized testing \cite{sirin}.  We intend to explore the relationship between other, more novel, aggregate measures of both SES and academic performance. Finally, we intend to provide an intuitive and interactive graphic interface to explore the results.

\section{Approach}
%What is new in your approach and why do you think it will be successful? 
Taking the graduation rate data from https://data.nysed.gov/, we intend to use a multiple models to characterize the relationship between academic performance and SES.  We will join any of the available relevant data sets and isolate the most important factors that affect academic performance at the school, neighborhood, district, and state level. We will also compare other factors like school types (e.g. charter schools) to socioeconomic factors.

We plan to use measures for academic performance including, but not limited to, aggregate test scores and aggregate graduation rates.  Then using D3 javascript library (D3), we will illustrate differences and allow the stakeholder to explore relationships using our interface. We seek to make correlation relationships, not causal relationships, as the data may not include all latent factors.

We will explore machine learning and statistical methods to find appropriate models.  For SES, we are using NRC (at the school / district level) and Perhaps County Historical Employment and Wages Data (at the county aggregate level) as features. We also plan to investigate confounding factors that could affect conclusions on SES, such as school types and teacher education.

We will also explore using fixed effect models as used in economic studies of charter school factors \cite{winters, jinnai}. Charter schools would be a confounding factor for SES effects. We may need to perform feature engineering to define the factors appropriately and must consider nuance as some factors like charter schools may only affect grades levels (e.g., 6th grade) that are overlapping \cite{jinnai}.

\section{Stakeholders}
%Who cares? If you are successful, what difference will it make? 
Some surprise beneficiaries of our research are people interested in healthier lifestyles for students, based on J.O. Lee's research \cite{lee}, which links poor academic performance with worse health outcomes. Additionally, researchers of the relationship between school climate, SES, and Academic performance would benefit; In Berkowitz's study of school climate impact on AP she noted a lack of consistent SES - AP modeling \cite{berkowitz}. Last, this project should be of interest to people invested in improving academic performance overall. While school administrators often point to overall improving academic performance \cite{domanico}, there is broad consensus that the SES - AP correlation is mitigated by well-resourced schools (such as charter schools) \cite{domanico, jinnai}. If we are successful, all of the above could use our research to help reinforce their arguments for improving resources for lower SES schools.

\section{Risks}
%What are the risks?
In Sirin, et. al.'s Review of Educational Research, one of the primary risks is 'ecological fallacy' \cite{sirin} -- a misinterpretation of the results where one applies aggregate results to individual outcomes.  This very much applies to our project, since we work with aggregate data at the school, district, county, and state level.  In order to not run afoul of this risk, we will need to clearly indicate, visually, what level we are operating at and reinforce that outcomes may not apply to individuals.

Another risk that data scientists should consider is the 'black box' problem of machine learning models. Gilpin claims that complex machines and algorithms cannot perform interpretation tasks \cite{gilpin}-- tasks better left for a human.  If our best models happen to be ML models that have many hidden layers (e.g. Deep Learning Convolutional Neural Networks) we run the risk of not being able to explain how the model arrives at its decisions.

Another risk is that the machine learning models might downplay important features if the features are highly correlated. We will need to examine feature correlation before making conclusions on feature importance.

% Additionally, in Chen's Study of School Violence, SES, and Student Achievement, he indicates that the study supports the model hypothesis that poverty and minority status predict school disorder, which relates to SES and Academic Performance \cite{chen}.   Scientifically, we want to include the features that inform our model the best.  Ethically, we want to be very cognizant of reinforcing stereotypes brought on by models that use race as a predictor.  To be frank, this is one of the complicated ethical questions of statistics and ML, and we will do our best to challenge each other that we are responsibly handling sensitive data. 

\section{Cost}
%How much will it cost? 
Based on How to Get a Free Lunch: A Simple Cost Model for Machine Learning Applications a straightforward cost model for ML applications can be applied.  This somewhat applies to our project, because while we hope to not incur additional equipment, storage, and compute costs, it is possible that our scope expands to where only distributed computing will be able to handle the calculations.  The cost for an ML project is the following:\\

$NPV = C_0 + \Sigma{\frac{C_t}{(1+r)^t}}$\\

Where $NPV$ is the net present value, $C_0$ is the initial cost of equipment, and $C_t$ is the decision cost. Since we are working with freely available data, and we intend to use compute \& storage we already own, the NPV of this project should be \$0.


\section{Time}
%How long will it take? 
In Kushal Singla, et. al's analysis \cite{mltime} of software engineering for Agile ML projects, they identify that a slight majority of tasks end up in the backlog for ML projects, and there is a huge time variance on how long a story takes. On average, an ML story takes about 18 hours to complete, with an extremely high standard deviation of 31 hours.  For our team of 5, we intend to divide and conquer many tasks with Nick and Dave working more heavily on the documentation, design, organization, presentation, and oversight; Vuong and Ben working more heavily on D3 visualization and python modeling, respectively; and Shitij working on a pretty even blend of all of the above.  We intend to adopt an agile approach, and remain flexible as new information / requirements arise.  We estimate 1 week for lit survey, proposal, and presentation development.  We estimate 5 stories (roughly 90±155h) in a two week sprint for data cleaning, loading, design, EDA, and preliminary modeling; another 5 (90±155h) stories for Statistics/ ML modeling, and 8 stories (144±248h) for D3 visualization, interactivity, testing and iteration.

\section{Success Criteria \& Evaluation}
%What are the mid-term and final “exams” to check for success? 
At a project level, our Minimum Viable Product (MVP) success criteria are a working, well-researched model and a working interactive data visualization that displays the results.  At a model level, success criteria would be a feature selection algorithm that works, with reasonable, explainable features.  We intend to evaluate the model using efficacy statistics that will vary, depending on what models we choose to compare.  At the bare minimum, we will use p-values to evaluate the efficacy of a regression model that may or may not include feature selection criteria.  At the most ambitions level, we will be able to enrich the NYS data with natural experiment data that would help us prove a causal (if aggregate) relationship.  There is, obviously, a lot of room in between those two goals for evaluation, and I suspect we'll fall nearer the former than the latter.  

We will know if we are successful if we can get data cleaned, model developed, and a Choropleth map working by the end of October as an MVP.

%P-values, F scores, working visualization \& MVP.  We are shooting for a minimum of two-model comparisons and a 40-60\% blend of modeling to visualization focus ratio.

\begin{thebibliography}{9}
\bibitem{jinnai}
Yusuke Jinnai
\textit{Direct and indirect impact of charter schools’ entry on traditional public schools: New evidence from North Carolina}
Economic Letters 124, Volume 124 (2014) 452-456

\bibitem{sirin}
Selcuk R. Sirin
\textit{Socioeconomic Status and Academic Achievement: A Meta-Analytic Review of Research }. 
Review of Educational Research, Fall 2005, Vol. 75, No. 3, pp. 417–453.

\bibitem{hearn}
James C. Hearn
\textit{Attendance at Higher-Cost Colleges: Ascribed, Socioeconomic, and Academic Influences on Student Enrollment Patterns}
Economics of Education Review, Vol. 7, No.  1, pp. 65-76, 1988.

\bibitem{farooq}
M.S. Farooq, A.H. Chaudhry, M. Shafiq, G. Berhanu
\textit{Factors Affecting Students' Quality of Academic Performance: A Case of Secondary School Level }.
Journal of Quality and Technology Management, Volume VII, Issue II, December, 2011, pp. 01-04

\bibitem{parentalinvolvement} 
Aprile D. Benner, Alaina E. Boyle, Sydney Sadler1
\textit{ Parental Involvement and Adolescents’ Educational Success: The Roles of Prior Achievement and Socioeconomic Status }. 
Springer Science+Business Media New York 2016

\bibitem{collegescores} 
Rebecca Zwick 
\textit{The Role of Admissions Test Scores, Socioeconomic Status, and High School Grades in Predicting College Achievement}. 
Pensamiento Educativo. Revista de Investigación Educacional Latinoamericana 2012, 49(2), 23-30

\bibitem{raceses} 
Juan Battle \& Michael Lewis
\textit{ The Increasing Significance of Class: The Relative Effects of Race and Socioeconomic Status on Academic Achievement }. 
Journal of Poverty, 6:2, 21-35, DOI: 10.1300/J134v06n02\textunderscore02

\bibitem{lee} 
J.O. Lee, R. Kosterman, T.M. Jones, T.I. Herrenkohl, I.C. Rhew, R.F. Catalano, J.D. Hawkins
\textit{ Mechanisms linking high school graduation to health disparities in young adulthood: a longitudinal analysis of the role of health behaviours, psychosocial stressors, and health insurance }. 
Public Health Vol. 139 (2016) pp. 61-69.

\bibitem{gilpin}
L. H. Gilpin, D. Bau, B. Z. Yuan, A. Bajwa, M. Specter and L. Kagal, 
\textit{Explaining Explanations: An Overview of Interpretability of Machine Learning}, 2018 IEEE 5th International Conference on Data Science and Advanced Analytics (DSAA), Turin, Italy, 2018, pp. 80-89, doi: 10.1109/DSAA.2018.00018.

% \bibitem{chen}
% Greg Chen \& Lynne A. Weikart (2008) 
% \textit{Student Background, School Climate, School Disorder, and Student Achievement: An Empirical Study of New York City's Middle Schools }. Journal of School Violence, 7:4, 3-20, DOI: 10.1080/15388220801973813

\bibitem{domanico}
Ray Domanico
\textit{NYC Student Achievement: What State and National Test Scores Reveal}, 
The Manhattan Institute, 2020

\bibitem{mltime}
K. Singla, J. Bose and C. Naik, 
\textit{Analysis of Software Engineering for Agile Machine Learning Projects }.
2018 15th IEEE India Council International Conference (INDICON), Coimbatore, India, 2018, pp. 1-5, doi: 10.1109/INDICON45594.2018.8987154.

\bibitem{berkowitz}
Ruth Berkowitz, Hadass Moore, Ron Avi Astor, Rami Benbenishty
\textit{A Research Synthesis of the Associations Between Socioeconomic Background, Inequality, School Climate, and Academic Achievement }.
Review of Educational Research April 2017, Vol. 87, No. 2, pp. 425–469 DOI: 10.3102/0034654316669821

\bibitem{winters}
Marcus Winters
\textit{Measuring the effect of charter schools on public school student achievement in an urban environment: Evidence from New York City}
Economics of Education Review 31, Volume 31 (2012) 293-301



\end{thebibliography}
% %%
% %% The acknowledgments section is defined using the "acks" environment
% %% (and NOT an unnumbered section). This ensures the proper
% %% identification of the section in the article metadata, and the
% %% consistent spelling of the heading.
% \begin{acks}
% To Robert, for the bagels and explaining CMYK and color spaces.
% \end{acks}

%%
%% The next two lines define the bibliography style to be used, and
%% the bibliography file.
\bibliographystyle{ACM-Reference-Format}
\bibliography{sample-base}

%%
%% If your work has an appendix, this is the place to put it.
\appendix

% \section{Research Methods}

% \subsection{Part One}

% Lorem ipsum dolor sit amet, consectetur adipiscing elit. Morbi
% malesuada, quam in pulvinar varius, metus nunc fermentum urna, id
% sollicitudin purus odio sit amet enim. Aliquam ullamcorper eu ipsum
% vel mollis. Curabitur quis dictum nisl. Phasellus vel semper risus, et
% lacinia dolor. Integer ultricies commodo sem nec semper.

% \subsection{Part Two}

% Etiam commodo feugiat nisl pulvinar pellentesque. Etiam auctor sodales
% ligula, non varius nibh pulvinar semper. Suspendisse nec lectus non
% ipsum convallis congue hendrerit vitae sapien. Donec at laoreet
% eros. Vivamus non purus placerat, scelerisque diam eu, cursus
% ante. Etiam aliquam tortor auctor efficitur mattis.

% \section{Online Resources}

% Nam id fermentum dui. Suspendisse sagittis tortor a nulla mollis, in
% pulvinar ex pretium. Sed interdum orci quis metus euismod, et sagittis
% enim maximus. Vestibulum gravida massa ut felis suscipit
% congue. Quisque mattis elit a risus ultrices commodo venenatis eget
% dui. Etiam sagittis eleifend elementum.

% Nam interdum magna at lectus dignissim, ac dignissim lorem
% rhoncus. Maecenas eu arcu ac neque placerat aliquam. Nunc pulvinar
% massa et mattis lacinia.

\end{document}
\endinput
%%
%% End of file `sample-sigconf.tex'.
